\begin{singlespace}
    \chapter{Introduction}
\end{singlespace}

\begin{itemize}
    \item At least 5 pages.
    \item Summary of the whole thesis. Use previous studies, diagrams, and illustrations to emphasize the motivation behind this thesis.
    \item Must answer the following:
    \begin{enumerate}
        \item What is the problem that I am solving and why does it matter?
        \item What are the state-of-the-art (SOTA) solutions to this problem?
        \item What is the gap in the current SOTA?
        \item What is your idea to address this gap?
        \item Why do you think your idea will work?
        \item How will you execute your idea?
        \item What is the empirical evidence that your idea works?
        \item What can you conclude from the study that you have accomplished?
        \item What are the possible future works that will extend your study?
    \end{enumerate}
    \item List the roadmap to the rest of the manuscript.
\end{itemize}

\textbf{Start Strong:} For every chapter (except possibly the Problem Statement), make an introduction (2 or 3 paragraphs) on what the chapter is all about.

\textbf{Stay Strong:} Explain ideas in the simplest and most direct way that many people in your field can understand. If a certain topic is a bit specialized or hard to remember, make a concise introduction. Point the reader to a reference for further understanding. Each chapter should be complete or stand-alone and concise.

\textbf{Finish Strong:} At the end, make a summary (2 or 3 paragraphs) to re-emphasize the points discussed in the chapter.

\section{Scope and Limitations}

What is the scope of your work? What are its limitations?

\section{Structure}

This thesis is organized as follows. In \Cref{ch:related-work}, the discussion on the body of work contextualizes our approach. \Cref{ch:problem-statement} discusses the problem statement of this thesis. In \Cref{ch:methodology}, the methodology is discussed in more detail. \Cref{ch:results} contains the evaluation results, while \Cref{ch:analysis} contains the analysis and discussion. The thesis is concluded in \Cref{ch:conclusion}.
