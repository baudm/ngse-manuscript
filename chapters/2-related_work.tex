\sschapter{Related Work}
\label{ch:related-work}

\begin{itemize}
    \item Expound \#2 (\emph{What are the state-of-the-art (SOTA) solutions to this problem?}) and \#3 (\emph{What is the gap in the current SOTA?}) by rigorously enumerating related works and analyzing these in the context of the problem that you are solving.
    \item Build a taxonomy or survey to narrow down the field of study of the problem and to limit the scope of your thesis. If there is a recent survey paper in your problem, use it. If none, use Google Scholar to build a tree diagram of related work.
    \item Build a table or graph with metrics to show what are available features and what are lacking in the current SOTA.
    \item Using the table/graph, identify the gap to show what do you intend to solve.
    \item Introduce the idea on how to solve this gap.
\end{itemize}

\section{Chapter Summary}

\textbf{Start Strong:} For every chapter (except possibly the Problem Statement), make an introduction (2 or 3 paragraphs) on what the chapter is all about.

\textbf{Stay Strong:} Explain ideas in the simplest and most direct way that many people in your field can understand. If a certain topic is a bit specialized or hard to remember, make a concise introduction. Point the reader to a reference for further understanding. Each chapter should be complete or stand-alone and concise.

\textbf{Finish Strong:} At the end, make a summary (2 or 3 paragraphs) to re-emphasize the points discussed in the chapter.
