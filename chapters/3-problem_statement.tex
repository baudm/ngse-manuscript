\chapter{Problem Statement}
\label{ch:problem-statement}

\begin{itemize}
    \item Following up from \#3 (\emph{What is the gap in the current SOTA?}), formalize the main problem and subproblems using a list.
    \item Use math models and diagrams to clearly show the problem and subproblems being addressed (\eg prior work uses $P(\mathbf{y} | \mathbf{y}_{<t}, \mathbf{x})$ as the model, while we use $P(\mathbf{y} | \mathbf{y}_{\neq t}, \mathbf{x})$).
\end{itemize}

\textbf{Start Strong:} For every chapter (except possibly the Problem Statement), make an introduction (2 or 3 paragraphs) on what the chapter is all about.

\textbf{Stay Strong:} Explain ideas in the simplest and most direct way that many people in your field can understand. If a certain topic is a bit specialized or hard to remember, make a concise introduction. Point the reader to a reference for further understanding. Each chapter should be complete or stand-alone and concise.

\textbf{Finish Strong:} At the end, make a summary (2 or 3 paragraphs) to re-emphasize the points discussed in the chapter.

\section{Objectives}

The specific objectives of this thesis are:
\begin{enumerate}
    \item Propose a new method X to solve problem Y.
    \item Compare methods A, B, and C against our method X.
\end{enumerate}
