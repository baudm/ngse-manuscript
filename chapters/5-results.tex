\begin{singlespace}
    \chapter{Results}
    \label{ch:results}
\end{singlespace}

\begin{itemize}
    \item Expound \#6 (\emph{How will you execute your idea?}) and \#7 (\emph{What is the empirical evidence that your idea works?}).
    \item Make a complete description of your experimental setup (\eg dataset, train and test/validation configurations, hardware configurations, software framework).
    \item Describe the metrics (performance measures) that are used to benchmark the task. These are the same metrics in the review of lit. Sometimes, you may need to introduce new metrics. However, you have to have a strong justification on why there is a need for a new metric and it is a good measure of performance in a task.
    \item Make sure the metrics are comprehensive (\eg include model parameter count, FLOPS, inference time, memory use, energy consumption, \etc).
    \item Use graphs and tables to summarize the quantitative results from your proposed method vs SOTA.
    \item Illustrate sample outputs to qualitative describe the results of your experiments.
\end{itemize}

\section{Chapter Summary}

\textbf{Start Strong:} For every chapter (except possibly the Problem Statement), make an introduction (2 or 3 paragraphs) on what the chapter is all about.

\textbf{Stay Strong:} Explain ideas in the simplest and most direct way that many people in your field can understand. If a certain topic is a bit specialized or hard to remember, make a concise introduction. Point the reader to a reference for further understanding. Each chapter should be complete or stand-alone and concise.

\textbf{Finish Strong:} At the end, make a summary (2 or 3 paragraphs) to re-emphasize the points discussed in the chapter.
