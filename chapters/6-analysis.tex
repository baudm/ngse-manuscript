\chapter{Analysis and Discussion}
\label{ch:analysis}

\begin{itemize}
    \item As a follow up of \#7 (\emph{What is the empirical evidence that your idea works?}), given the experimental results, make an in-depth analysis and discussion to argue and justify that your proposed method works.
    \item Use further evidences (\eg attention maps) to emphasize the strong points of your method.
    \item Make an ablation study (\eg what if we vary the depth of the network, what if we introduce data corruption, etc) to further show strong/weak points of your proposed method.
    \item No method is perfect. What are the failure cases of your method (\eg method does not work on rotated text). Explain why your method fails in these cases.
\end{itemize}

\section{Chapter Summary}

\textbf{Start Strong:} For every chapter (except possibly the Problem Statement), make an introduction (2 or 3 paragraphs) on what the chapter is all about.

\textbf{Stay Strong:} Explain ideas in the simplest and most direct way that many people in your field can understand. If a certain topic is a bit specialized or hard to remember, make a concise introduction. Point the reader to a reference for further understanding. Each chapter should be complete or stand-alone and concise.

\textbf{Finish Strong:} At the end, make a summary (2 or 3 paragraphs) to re-emphasize the points discussed in the chapter.
